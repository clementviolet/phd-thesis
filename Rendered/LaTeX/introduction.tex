\hypertarget{introduction}{%
\chapter*{Introduction}\label{introduction}}
\addcontentsline{toc}{chapter}{Introduction}

\chaptermark{Introduction}

\hypertarget{les-uxe9costysuxe8mes-cuxf4tiers}{%
\section*{Les écostysèmes
côtiers}\label{les-uxe9costysuxe8mes-cuxf4tiers}}
\addcontentsline{toc}{section}{Les écostysèmes côtiers}

La notion d'habitat est essentielle en écologie. Cette notion englobe
non seulement le lieu où une espèce peut être trouvée, le(s) lieu(x) où
elle réalise son cycle de vie, mais également l'ensemble des conditions
abiotiques et biotiques qui permettent le maintien de sa population
\autocite{Hall_1997}. Certaines espèces animales ou végétales ont la
capacité de modifier leur environnement de nombreuses façons
différentes, comme en modifiant les conditions environnementales
\autocite{Ellison_2019} ou en produisant des structures
tridimensionnelles complexes permettant d'abriter d'autres espèces
\autocite{Darling_2017}. Ces espèces sont qualifiées d'espèces
ingénieurs \autocite{Jones_1996}. Ces espèces ingénieurs peuvent de par
leur seule présence caractériser un habitat, c'est pourquoi dans la
suite de ce manuscrit, nous allons qualifier les habitats créés par ces
espèces d'habitats biogéniques.

Les habitats biogéniques structurent la dynamique des écosystèmes
benthiques côtiers auxquels elles sont associées \autocites[
]{Duffy_2006}{Teagle_2017}. Ces habitats vont promouvoir une plus grande
diversité spécifique au sein des écosystèmes qui les abritent
\autocites[ ]{Romero_2015}{Sunday_2016}, notamment de par leur capacité
à tamponner les conditions environnementales locales en modifiant par
exemple la température \autocite{Bulleri_2018}. A une échelle locale,
les habitats biogéniques tendent donc à augmenter l'hétérogénéité
spatiale de l'environnement et par conséquent augmentent la diversité et
l'abondance de niches environnementales disponibles pour d'autres
espèces \autocites[ ]{Duffy_2006}{Hewitt_2005}. Ces modifications
locales de l'environnement par les habitats biogéniques entretiennent
des boucles de rétroactions négatives \autocite{Kefi_2016}, ces boucles
de rétroactions négatives ont des effets stabilisateurs des conditions
environnementales, ce qui à une échelle globale augmente la résilience
des écosystèmes face aux perturbations entraînées par le changement
climatique global (fig.~\ref{fig:1}; \textcite{Bulleri_2018} ;
\textcite{Jurgens_2022}). Un exemple simplifié est celui des herbiers
marins, qui assurent la protection des mezobrouteur\footnote{Les
  mezobrouteurs sont de petits invertébrés herbivores d'une taille
  inférieure ou égale à 2,5 cm \autocite{Beermann_2018}.} contre leurs
prédateurs, qui vont se nourrir des épiphytes permettant à l'herbier de
se maintenir et de s'agrandir. Ainsi, ces boucles de rétroactions
négatives ont des effets stabilisateurs des conditions
environnementales, ce qui à une échelle globale augmente la résilience
des écosystèmes face aux perturbations entraînées par le changement
climatique global \autocites[ ]{Bulleri_2018}{Jurgens_2022}.

La promotion de la biodiversité par les habitats biogéniques a également
un impact sur les services écosystémiques fournis par les écosystèmes
marins. Ces habitats biogéniques sont notamment des zones importantes
pour le cycle de vie de poissons à forte valeur commerciale. Par
exemple, les prés-salés et les herbiers marins sont des zones de
nourricerie essentielles au développement des juvéniles comme ceux du
bar commun (\emph{Dicentrarchus labrax}) \autocites[
]{Stamp_2022}{Maxwell_2017}. Ces habitats biogéniques ont également un
fort impact économique ; les récifs coralliens supportent par exemple
une industrie du tourisme conséquente (valeur estimée à près de
36~milliards de dollars par an ; \textcite{Spalding_2017}). Enfin, les
habitats biogéniques jouent un rôle particulièrement important dans la
lutte contre l'érosion, via leur effet de stabilisation du sédiment
\autocite{Reidenbach_2018} ou bien en réduisant l'énergie des vagues
lors d'évènements extrêmes comme les tempêtes \autocite{Krauss_2019} ou
bien les tsunamis \autocite{Alongi_2008}.

Malgré leur importance, les écosystèmes côtiers font face à de
nombreuses menaces comme l'acidification des océans
\autocite{Doney_2009}, l'augmentation des températures
\autocite{IPCC_2021_technical_summary}, l'augmentation du nombre et de
la fréquence des évènements climatiques extrêmes \autocite{Oliver_2018},
ainsi que de leur surexploitation \autocites[ ]{Lotze_2006}{Lotze_2009}.
Ainsi, les habitats biogéniques qui sont de nature souvent fragile, car
formés par des espèces souvent sensibles aux différents impacts
anthropiquesont et ont décliné tant en qualité (i.e.~dégradation de
l'état écologique, ou perte de complexité, etc.), qu'en quantité
(i.e.~souvent exprimé comme une réduction de leur superficie) à travers
l'Europe et dans le monde entier \autocites[ ]{Airoldi_2007}[
]{Dunic_2021}[ ]{McCauley_2015}{Waycott_2009}. Ces pertes des habitats
biogéniques sont l'un des principaux moteurs du déclin de la
biodiversité auquel nous assistons actuellement \autocites[
]{Airoldi_2007}[ ]{ipbes_2019}{McCauley_2015}.

Ces différentes pressions d'origine anthropique et leurs conséquences
sur les habitats biogéniques vont affecter par effet de cascade
fonctionnement et dynamique globale des écosystèmes côtiers \autocites[
]{Rocha_2015a}[ ]{Sara_2021}{Wernberg_2016}. Les écosystèmes étant des
objets dynamiques, ils peuvent répondre de différente manière à ces
pressions. Après avoir dépassé leur seuil de résistance envers ces
perturbations, ils vont être transformés de façon linéaire, ou bien de
manière non linéaire. Ces changements non linéaires entraînent des
modifications profondes des écosystèmes appelés ``changement de régime''
\autocite{Scheffer_2001}. Dans le cas des transitions linéaires, il
serait possible d'identifier des états d'habitats différents de l'état
pristin plus ou moins dégradé \autocite{Spake_2022}. Lorsque les
transformations suivent des tendances non linéaires, l'apparition de
seuils empêchera le retour à l'état pristin des écosystèmes
\autocite{Spake_2022}.

Ces changements de régimes et d'états ont d'ores et déjà été observés à
travers un grand nombre d'écosystèmes terrestres et marins, par exemple
: la transition entre la toundra et la forêt boréale
\autocite{Folke_2004}, entre la forêt et la savane \autocites[
]{Debra_2004}{Folke_2004}, mais également, comme pour les forêts de
laminaires qui peuvent laisser place à des déserts d'oursins \autocites[
]{Carnell_2020}{Rogers-Bennett_2019}, ou bien encore les récifs
coralliens supplantés par des macroalgues \autocites[
]{Folke_2004}{OBrien_2018}. Le déclin des habitats biogénique est
notamment l'un des moteurs des changements de régimes observés ces
dernières décennies \autocites[ ]{Rocha_2015a}{Wernberg_2016}.

Pour éviter la dégradation des écosystèmes marins liée à la disparition
des habitats biogéniques, les écologues font face à de nombreux défis à
relever. Un premier défi consiste à identifier et décrire les différents
types d'habitats biogéniques à l'échelle mondiale de façon adéquate pour
identifier les changements d'état d'habitat. En effet, les typologies
existantes ont du mal à détecter ces changements \autocite{Cooper_2019},
ce qui entrave la mise en place de mesures de gestion efficaces
\autocite{Ware_2020}. Par conséquent, il est essentiel de développer une
typologie d'habitat mondiale basée sur les suivis biologiques existants
\autocite{Cooper_2019} afin de mieux coordonner les politiques de
préservation des habitats côtiers \autocite{Ware_2020}. Le second défi
réside dans la prévention des changements d'état d'habitats biogéniques.
Pour ce faire, il est nécessaire d'identifier les facteurs externes et
internes qui contribuent d'une part à leur résilience et d'autre part
qui facilitent les transitions vers d'autres types d'habitats. Grâce à
une meilleure connaissance de ces facteurs, il sera possible
d'identifier des zones géographiques plus à risque de changement de
régime, et donc de mettre en place des mesures de suivi adéquat
permettant de réagir dès les premiers signes de changement de régime
observés.

Le développement de nouveaux outils méthodologiques de modélisation peut
permettre de faire face au premier défi. Pour le second, deux approches
différentes existent : une expérimentale, grâce à des expérimentations
\emph{in situ} ou en méscosome et une approche de modélisation. Ces
expérimentations \emph{in situ} ou en mésocosme présentent certains
avantages comme celui d'avoir une observation directe des phénomènes
étudiés \autocite{Fulton_2019}. Cependant, elles sont souvent réalisées
à des échelles spatiales restreintes et sur des périodes de temps
limitées, particulièrement lorsque les études s'intéressent aux
écosystèmes marins du fait de leurs contraintes \autocite{Witman_2015}.
Ainsi, les interactions entre les organismes, les conditions
environnementales et les pressions observées lors de ces
expérimentations peuvent être différentes de ce qui passe à plus large
échelle spatiale et/ou temporelle. Les résultats de ces expérimentations
peuvent alors être plus difficilement généralisables
\autocite{Witman_2015}.

Pour surmonter ces contraintes, l'utilisation d'outils de modélisation
peut se révéler utile. Les modèles sont des représentations simplifiées
des systèmes écologiques qui permettent leur étude à des échelles
spatiotemporelles plus grandes. Les modèles en écologie peuvent être
regroupés en deux catégories : de manière mécanistique ou corrélatives
\autocite{Kearney_2010}. Ces modèles peuvent être complexifiés à loisir
selon les hypothèses de recherche pour mieux prendre en compte la
complexité réelle des écosystèmes \autocite{Cartwright_2016}. Les
modèles permettent également de prédire les changements futurs des
habitats biogéniques en fonction des scénarios de changement climatique
\autocite{Curd_2023}.

L'objectif principal de cette thèse est d'améliorer la compréhension et
la prévision des changements potentiels dans les habitats biogéniques
des environnements benthiques côtiers. Pour cela, elle se concentre sur
l'identification des types d'habitats benthiques, l'exploration de leurs
états alternatifs et l'analyse des facteurs favorisant l'établissement
de ces différents états. Afin d'atteindre cet objectif, la thèse
s'appuie sur les données collectées par le programme \emph{Reef Life
Survey}, qui a effectué depuis 2008 des suivis des habitats et de la
faune des récifs côtiers à travers le monde. Grâce à ces données, la
thèse vise à identifier les principaux types d'habitats benthiques
côtiers en développant de nouvelles méthodes pouvant être adaptées à
d'autres écosystèmes. Elle contribue également à acquérir de nouvelles
connaissances sur les facteurs environnementaux et les pressions
d'origine humaine qui influent sur le maintien ou le changement de ces
habitats biogéniques.

Ce travail de thèse est découpé en trois axes majeurs qui s'attèlent à
répondre à ces différentes questions.

\begin{enumerate}
\def\labelenumi{\arabic{enumi}.}
\tightlist
\item
  Définition d'une typologie d'habitats côtiers à l'échelle du globe à
  partir des données \emph{in situ} issues du programme \emph{Reef Life
  Survey}.
\end{enumerate}

Le but de cet axe est de développer une méthodologie d'identification de
typologie d'habitat à partir de données de substrat fournies par le
programme de suivis des communautés de récifs \emph{Reef Life Survey}
\autocite{Edgar_2004}.

Le \emph{Reef Life Survey} recense grâce à des plongeurs les poissons
nageant en pleine eau, au niveau du récif sur un transect de 50m de long
et de 10m de large. Puis, un second passage est effectué le long du
transect pour dénombrer la faune cryptique et benthique associée au
récif. Pour cette seconde passe, la largeur du transect est réduite à 4m
(fig.~\ref{fig:2}). Enfin, lors d'un troisième passage, 20 quadrats sont
photographiés tous les 2,5m \autocite{Edgar_2020}. Les pourcentages de
couverture de faune et flore benthique sont ensuite évalués visuellement
grâce à l'outil d'annotation d'image \emph{Squidle+}
(\url{https://squidle.org/}).

Une nouvelle méthodologie de groupement adaptée à ces données a été
développée pour créer une typologie des habitats benthiques. Elle est
présentée, justifiée en détail dans l'Annexe A et sera bientôt envoyée à
une revue.

\begin{enumerate}
\def\labelenumi{\arabic{enumi}.}
\setcounter{enumi}{1}
\tightlist
\item
  Distribution des habitats côtiers \& identification des zones à risque
  de changements de types d'habitats
\end{enumerate}

La seconde partie de cette thèse s'intéressera à modéliser les niches
environnementales de chacun des types d'habitats précédemment découverts
pour (1) identifier les seuils environnementaux qui favorisent
l'occurrence de ces habitats (2) identifier les zones capables d'abriter
plusieurs types d'habitats (3) définir des habitats alternatifs en
étudiant le chevauchement de leurs niches environnementales. Un stage de
M2 a déjà été encadré sur cette thématique pour réaliser une première
exploration de ces questions de recherche.

L'un des sujets de discussion que le doctorant souhaite aborder lors de
ce CSI concerne l'identification d'une stratégie de modélisation
efficace pour répondre aux questions de recherche pendant la période
restante de son contrat de thèse.

\begin{enumerate}
\def\labelenumi{\arabic{enumi}.}
\setcounter{enumi}{2}
\tightlist
\item
  Potentiel des \emph{jSDM} pour l'étude des impacts des changements
  d'habitats sur les communautés : une perspective préliminaire
\end{enumerate}

Les habitats biogéniques jouent un rôle crucial dans les écosystèmes
côtiers en façonnant les communautés animales. Les perturbations
d'origine humaine perturbent l'état écologique de ces habitats et ont un
impact significatif sur les communautés associées en modifiant à la fois
les habitats des espèces et les conditions environnementales. La
présence d'une espèce dans un lieu et à un moment donné est limitée par
les conditions environnementales et les interactions avec d'autres
espèces, qui forment collectivement la niche environnementale de
l'espèce au sein de la communauté. Pour comprendre et anticiper les
conséquences des modifications de l'habitat sur le fonctionnement des
écosystèmes, il est essentiel de mener des études détaillées sur les
niches environnementales des espèces, englobant les facteurs abiotiques
et biotiques, ainsi que l'effet de l'habitat.

Cependant, en raison de la forte interdépendance entre l'habitat,
l'environnement et les interactions entre espèces dictées par la niche
environnementale, la compréhension des conséquences de la modification
de l'habitat nécessite de démêler les effets individuels de chaque
facteur. Des outils de modélisation prometteurs, tels que les modèles de
distribution d'espèces conjointe (\emph{Joint Species Distribution
Models} ou \emph{jSDM}), sont récemment apparus comme des solutions
potentielles à ce défi. Néanmoins, il y a actuellement un manque de
recul pour mettre en oeuvre de manière efficiente ces outils complexes
et exigeants, notamment en ce qui concerne leurs capacités explicatives
et prédictives, ainsi que leurs possibilités d'interprétation. C'est
pourquoi, avant de pouvoir être déployé sur un cas d'étude complexe, il
est nécessaire de comprendre le comportement de ces outils sur un jeu de
données connu pour évaluer l'impact des choix de modélisation sur les
performances d'un \emph{jSDM}. Ce travail a fait l'objet d'une
publication soumise au journal \emph{Methods in Ecology and Evolution}.
L'article soumis est présenté dans l'Annexe C.
