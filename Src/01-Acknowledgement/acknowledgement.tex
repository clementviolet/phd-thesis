\begin{refsection}

\hypertarget{remerciements}{%
\chapter*{Remerciements}\label{remerciements}}

Il y a un peu plus de vingt ans, qui aurait pu imaginer que le petit
garçon que j'étais, blotti dans sa serviette de plage, terrifié à la
seule pensée de fouler ce sable répugnant, souillé par les laisses de
mer et infesté de ces immondes amphipodes bondissant de toute part, se
destinerait à rédiger une thèse en écologie marine ? Ces quelques lignes
ne pourront exprimer toute la gratitude que j'ai envers vous, vous qui
m'avez permis d'arriver jusqu'au bout de ce périple.

Je souhaite tout d'abord exprimer ma sincère gratitude à Sakina-Dorothée
Ayata, Emmanuelle Cam, Thibaut de Bettignies d'avoir accepté de faire
partie de ce jury de thèse. Je souhaite remercier tout particulièrement
Stéphane Dray et David Mouillot pour avoir accepté d'en être les
rapporteurs.

Je remercie également Anik Brind'Amour, Philippe Cugier, Clément Garcia
Cédric Gaucherel, Olivier Gauthier pour avoir eu la gentillesse de faire
partie de mon CSI. Merci infiniment pour votre temps et votre écoute
attentive. Merci pour vos conseils tout au long de cette thèse, qui en
ont grandement amélioré la qualité.

Merci à Stanislas Dubois d'avoir dirigé cette thèse et pour l'écoute et
tout le temps que tu m'as accordé durant toute cette thèse.

Je souhaite exprimer ma profonde gratitude à mes directeurs de thèse,
débutant par toi, Martin, pour avoir accepté de m'accueillir sous ton
aile bienveillante dès le Master. Mes remerciements s'étendent également
à Aurélien et Mathieu pour leur participation essentielle dans cette
formidable aventure qu'a constituée l'encadrement de ma thèse. Votre
curiosité intellectuelle insatiable et votre rigueur m'ont constamment
poussé vers l'excellence, enrichissant indéniablement mon parcours. La
gentillesse et la bienveillance que vous avez tous trois manifestées à
mon égard ont grandement contribué au plaisir que j'éprouvais à me
rendre au laboratoire chaque jour.

Je tiens à exprimer ma profonde gratitude à tous ceux qui ont contribué
et qui contribuent encore au programme de suivi REBENT. Sans votre
travail assidu et votre dévouement, cette thèse n'aurait pas vu le jour.
I would also like to extend heartfelt thanks to Graham Edgar, Lizzi Oh,
and Rick Stuart-Smith for your invaluable assistance throughout my PhD
journey. Your dedication, alongside the tremendous efforts of all the
volunteers involved in the Reef Life Survey Program, has been
instrumental in the progression and success of my research. Your
contributions have not only enriched my work but also underscored the
profound impact of collaborative endeavour.

Je tiens à exprimer toute ma reconnaissance à Olivier Gauthier et
Emmanuelle Cam qui m'ont généreusement donné l'opportunité d'enseigner
durant ces trois années. Chaque heure passée en compagnie des étudiants
a été pour moi une source de joie intense, me poussant à approfondir
avec rigueur les sujets que j'avais la responsabilité de leur
transmettre.

Ma passion pour l'enseignement, je la dois en grande partie à toi,
Olivier. Tes cours n'ont pas été seulement instructifs ; ils ont été une
véritable révélation. Cependant, cette vocation ne s'est pas nourrie de
ta seule influence, aussi significative soit-elle (et c'est bien dommage
!). Je suis également redevable à Gauthier Schaal et Jacques Grall, qui
m'ont montré qu'enseigner pouvait se faire dans une ambiance chaleureuse
et décontractée. Enfin, mes sincères remerciements vont à Christophe et
David. Votre approche innovante et particulière de l'enseignement a
profondément transformé ma conception de la pédagogie, et pour cela, ma
gratitude est infinie.

Je souhaite du plus profond de mon coeur remercier Emilien, Touria,
Aline B., Aurélien B., Antoine, Mathieu, Céline, Philippe, Amelia,
Gabin, Stan, Justine, Aline G., Pierre-Olivier, Martin, Flavia, Aurélien
T. et Mick' pour faire du LEBCO et de sa salle de pause un endroit si
chaleureux, merci infiniment pour ces fous rires.

Un immense merci à Véronique pour avoir aidé durant ces trois années à
naviguer dans les différents arcanes administratifs d'Ifremer et de
m'avoir aidé à préparé mes différentes missions.

Un énorme merci à tous les doctorants (et les pièces rapportées !) de
l'Open Space pour cette ambiance incroyable qui nous booste chaque jour
! Une pensée spéciale pour ceux qui ont déjà pris leur envol vers de
nouvelles aventures : Bastien, Kévin, Lyndsay, Marion, Lou, et Alex.
Léa, dont la gentillesse et le rire ont le pouvoir magique de réchauffer
l'Open Space, même lors des matinées frisquettes d'automne, d'hiver, de
printemps ou d'été. Un p'tit clin d'oeil à notre Dinde Supérieure,
Laure, doyenne incontestée de notre petit univers ; chacune de nos
discussions a été un vrai kif ! Et toi, Mathisse, tu es simplement la
meilleure (et pas seulement pour le café brûlant qui m'accueille chaque
matin). Impossible de résumer ici tous ces moments épiques vécus
ensemble, que ce soit en installant du carrelage et des moulures dans le
bureau de Mathieu, lors des sessions cours de surf ou lors de nos
innombrables soirées bars. En parlant bar toi, Bastouille, avec tes
exploits légendaires, tu restes indétrônable ! Nos disucssions Tour de
France et les étapes dans le Pays-Basque resteront gravés dans ma
mémoire. Un immense merci également, pour ces moments ``oxygénation''
ces dernières semaines, j'en avais grandement besoin. Mathieu, tu es
sans conteste LA `pièce rapportée' de cet Open Space, merci pour ces
discussions et ces fous rires en pause ou à la cantine qui resteront
gravés dans ma mémoire ! Chonchon, un merci tout spécial pour ton super
boulot durant ton M2, essentiel pour ma thèse, et merci pour tous ces
mèmes politiques croustillants sur Twitter, quelle rigolade ! Merci,
Irene, d'avoir enduré mon espagnol hésitant sans jamais perdre patience,
c'est un exploit ! Quant aux petits nouveaux, Lucas, Maeva, et Chloé,
votre aventure ne fait que commencer, mais je suis déjà ravi d'avoir
croisé votre route. Je suis persuadé que vous allez faire un excellent
job. Savourez chaque instant, malgré les défis que la thèse représente,
ça en vaut la peine ! Un immense merci à tous pour ces moments
inoubliables !

Un phénoménal big-up à mon crew, la team ``A Fond la Fonte'' : Manon,
Marine, Thomas, Mathisse, Laure, Aurélien, Mathieu ! J'avoue, mon
assiduité brillait peut-être plus à la cantine qu'à la salle de sport.
Mais qu'importe, chaque déjeuner passé ensemble était épique. Merci pour
ces instants mémorables !

Je tiens à saisir cette opportunité pour exprimer ma profonde gratitude
envers tous mes amis qui ont jalonné mon parcours depuis mon arrivée à
Brest. Un merci tout particulier à Elodie, Elodie, Pierre, Thibault,
François, Anthony, Mathilde, Angelina, Maurane, Christophe et David pour
tous ces précieux moments partagés. Ma reconnaissance s'étend également
aux amis rencontrés au cours de ce voyage académique, notamment Lyndsay,
Léa, Lou, Kévin, Lucas, Pierre-Léo, Aurélien, Gabin, Mathieu et Noémie,
Romain, Fred, Gaspard et Claire, qui ont chacun illuminé ce voyage.

Merci au Tortuga, aux Fauvettes et maintenant à Gourmand, Mais Pas Que
(a.k.a. l'After Work) d'avoir été les partenaires officiels de ma thèse.

Je souhaite rendre un hommage tout particulier à Masha, mon tigre
domestique, qui a fait de moi son humain de compagnie.

Merci infiniment à ma famille, et surtout à mes parents, vous méritez
amplement toute ma gratitude pour votre soutien constant tout au long de
mes études, une fondation solide qui m'a porté jusqu'à cet instant où je
rédige ces quelques phrases. Sans vous, rien de tout cela n'aurait été
possible et je vous en suis infiniment reconnaissant. Une petite mention
spéciale à toi Guillaume, pour avoir partagé ces années inoubliables à
Brest avec moi, affichant avec fierté les couleurs du Stade Brestois 29.
Ensemble, nous avons bravé la pluie, le vent, la neige, ou même les
trois à la fois, dans l'arène légendaire du stade Francis-Le Blé. Merci
frangin !

Je souhaite également remercier ma belle famille. Gilles, Dominique,
Gwen (sans oublier le p'tit Maël), un merci du fond du cœur pour votre
accueil chaleureux, ainsi que pour le soutien indéfectible et
l'affection que vous m'avez généreusement offerte au fil des années.

Gaëlle, ton amour et ton soutien inconditionnel ont été mon refuge et ma
force durant cette thèse. Ces années n'auraient pas eu la même saveur
sans ta présence constante et rassurante à mes côtés. Je sais que les
derniers mois ont été rudes pour toi. Sans toi, ce travail de thèse
n'aurait pas seulement été plus difficile --- il aurait été impossible.
Je t'aime.

Ces remerciements sont déjà beaucoup trop longs, mais sache que je te
remercie du font du coeur toi, toi que je n'ai malheureusement pas eu la
place de citer, mais qui a croisé ma route pendant cette aventure
universitaire et l'a rendu unique. Merci infiniment.

\printbibliography[heading=subbibintoc, title={Bibliographie}]
\end{refsection}