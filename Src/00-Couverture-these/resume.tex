\markboth{}{}
% Plus petite marge du bas pour la quatrième de couverture
% Shorter bottom margin for the back cover
\newgeometry{inner=30mm,outer=20mm,top=40mm,bottom=20mm}

%insertion de l'image de fond du dos (resume)
%background image for resume (back)
\backcoverheader

% Switch font style to back cover style
\selectfontbackcover{ % Font style change is limited to this page using braces, just in case

\titleFR{Approches quantitatives pour comprendre et prédire l’écologie, la distribution et la biodiversité des habitats benthiques dans l’Anthropocène}

\keywordsFR{Ecologie des communautés, Ecologie numérique, Habitats benthiques, Modélisation, Anthropocène}

% 200 words max
\abstractFR{L’objectif de cette thèse est de mieux comprendre et prédire la biodiversité benthique et le rôle des habitats biogéniques, dans le maintien de la structure et des fonctions des écosystèmes côtiers. Cette thèse a exploré différents outils numériques et des pipelines innovants et complémentaires pour répondre à ces objectifs à différentes échelles : 1) la modélisation jointe de la distribution des espèces dans deux habitats biogéniques à une échelle régionale et 2) la définition et la modélisation, via des approches de Machine Learning, de la distribution de l’état d’habitats benthiques à une échelle globale et nationale. Ces approches complémentaires contribuent à une meilleure quantification   de l’influence relative des facteurs environnementaux et anthropiques (notamment épisodes de canicules marines et pression de pêche) qui déterminent la biodiversité côtière et l’état des habitats benthiques.  Si dans les deux cas d’étude, la prédictibilité des espèces considérées ou des états était faible, ces travaux ont mis en évidence des stratégies pour optimiser l’inférence et la prédiction des modèles explorés. Ainsi, cette thèse apporte un point de vue critique sur les approches permettant d'étudier et de caractériser la biodiversité côtière, et sur les développements nécessaires pour mieux anticiper les réponses écologiques futures liées aux impacts anthropiques.
}

\titleEN{Quantitative approaches to understand and predict the ecology, distribution and biodiversity of benthic habitats in the Anthropocene}

\keywordsEN{Community ecology, Numerical ecology, Benthic habitats, Modelling, Anthropocene}

% 200 words max
\abstractEN{This thesis aims at better understanding and predicting coastal benthic biodiversity with a specific focus on the role of biogenic habitats in maintaining ecosystem  structure and functioning. This thesis explored how different innovative and complementary numeric tools and pipelines can address these objectives at different scales: 1) joint species distribution modelling  across two biogenic habitats at a regional scale, and 2) using Machine Learning approaches, defining and modelling the distribution of benthic habitats states at a global and at a national scale. These complementary approaches quantify the relative influence of the environmental and anthropogenic factors (including marine heatwaves and fishing intensity) that determine coastal biodiversity and the state of benthic habitats. While in both case studies the predictability of the considered species or states was low, these studies have identified future avenues to optimise models  inference and prediction of benthic communities. Thus, this thesis provides a critical perspective on existing approaches available to  study and characterise coastal biodiversity; and on the future developments required to better anticipate future ecological responses related to anthropogenic impacts.}

}

% Rétablit les marges d'origines
% Restore original margin settings
\restoregeometry
